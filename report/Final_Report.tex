\documentclass{article}
\usepackage{geometry}
\usepackage{graphicx}
\usepackage{hyperref}
\usepackage{enumitem}
\usepackage{amsmath}
\usepackage{fancyhdr}
\geometry{a4paper, margin=1in}


\title{Final Report: Morse Code Receiver System -- Group 4 \\ \large Microcontroller Based Systems (CCE2014)}
\author{Ignacy Manturewicz, Bradley Tabone, Aidan Cristina, Biniam Okubamicael}
\date{\today}

\pagestyle{fancy}
\fancyhf{}
\rhead{Group 4 -- Morse Code Receiver}
\lhead{Final Report}
\cfoot{\thepage}

\begin{document}

\maketitle

\begin{abstract}
This report explains the final implementation of a Morse code receiver system using the
 LPC4088 microcontroller. The system is designed to decode audio-based Morse code signals 
 and display translated characters on an LCD screen in real time. 
 It includes analog signal processing, filtering, demodulation, Morse decoding, 
 and visual error handling.
\end{abstract}

\section{Introduction}
    The goal of this project was to design and build a working Morse code receiver that could detect, decode, 
and display Morse audio messages. This involved developing a complete system that handles analog signal processing, 
digital demodulation, and real-time character output.

The implementation was carried out on the Embedded Artists’ LPC4088 Experiment Board, which allowed the use of
the microcontroller’s ADC, GPIO, and LCD capabilities. Combining these hardware features 
with a C program, resulted in turning the raw analog audio into readable text.

The project was approached step by step, starting with hardware setup (DC biasing circuit) and 
gradually expanding to comply with features like demodulation, decoding, and LCD output. In the end, everything
was brought together into a complete and working system. 
The project helped with gaining hands-on experience with microcontroller based systems and showed
how different parts of a real-time application come together.

\section{System Architecture}

\subsection{Hardware Configuration}
The system was built using:
\begin{itemize}
    \item A laptop to play audio Morse code signals.
    \item A breadboard with a DC-biasing circuit to center the signal around \textbf{2V}.
    \item A buzzer to retransmit the input signal.
    \item The LPC4088 microcontroller for sampling, processing, and decoding.
    \item A red LED used to show when the decoded message includes invalid characters.
    \item A 2-line LCD to display the final output.
    \item A push-button for reset.
    \item A switch used for clearing the LCD. 
\end{itemize}

\subsection{Software Design}
The C code was written and compiled using Keil µVision. The software includes:
\begin{itemize}
    \item Sampling the signal using ADC input.
    \item Filtering the samples using moving average.
    \item Demodulating the signal.
    \item Decoding Morse sequences using a lookup table.
    \item Displaying decoded characters on the LCD.
\end{itemize}


\section{Implementation Overview}

\subsection{Signal Acquisition and Preprocessing}
First, the signal is obtained from the laptop's mini-jack audio output and transferred to a breadboard 
using crocodile clips. It then passes through a DC biasing circuit (centered around 2V) 
before being routed to the ADC input pin of the experimental bundle via jumper wires.
The signal also goes through a buzzer, which is used to retransmit the input signal.

\subsection{Demodulation and Decoding Logic} 
The process begins with continuous sampling of the input rectified through the ADC on the 
LPC4088 microcontroller. Instead of relying on a single reading,
a moving average filter is applied over multiple consecutive samples to filter out noise and smooth out
fluctuations.
 The filtered output is then compared against a fixed threshold to determine 
whether the signal is currently active (tone) or inactive (silence).
 If the signal is above this threshold, a tone is considered present, and a \texttt{tone\_duration}
  counter is incremented. Conversely, if the signal drops below the threshold, the system interprets 
  it as silence, and a \texttt{silence\_duration} counter increments. 

When the tone ends, its duration is used to classify it either as a dot (\texttt{.}) 
or as a dash (\texttt{-}). Dots and dashes are stored sequentially in a buffer, 
building up each Morse symbol. Once the silence duration reaches a value signifying 
either the end of a character or a word, the buffer is finalized and passed to a lookup function.

This function compares the constructed Morse sequence with a predefined lookup table
(compliant with ITU-R standard) and returns the corresponding character, or a `\#` if it is unrecognized. 

\subsection{Character Display and Output Management}
Characters are printed on the LCD in real time. To avoid overflow, 
the most recent 16 characters are shown, the older ones scroll off screen. 
The system also shows live Morse signals (dots and dashes) on the first LCD row.
To handle word separation, the spaces are inserted in the message when a long silence is detected. 
If the character displayed is a `\#` then a red LED is turned on. Additionally, if a switch is pressed 
the display is cleared. When the signal stops, the system starts waiting for a new signal.

\subsection{Functionality overview}
The system is set to decode Morse code signals at a rate of 50 words per minute. It accepts tones with
frequencies between 300Hz and 800Hz. The frequency of the signal can be changed during runtime
and the system will adapt to it automatically. The system is also resilient to noise. However,
the precise timing of the tones is crucial for accurate decoding.




\section{Functional Requirements Review}
\begin{itemize}
    \item The system displays the digits while receiving the signal.
    \item The switch is used to clear the LCD.
    \item The LED turns on when there is an error in the message.
    \item System displays the message on the LCD.
    \item After power up, system is stable and waits for the signal.
    \item System configuration is persistent between reboots.
    \item Error conditions are indicated by the LED.
    \item System handles incorrect input without crashing.
    \item The microcontroller enters sleep mode between taking samples.
    \item System uses an interrupt handler which also suffices for multi-threading.
\end{itemize}


\section{Testing, Challenges and Improvements}
Throughout the project, several technical challenges were encountered that improved authors'
understanding of the system and therefore the implementation improved:

\begin{itemize}

     
    \item \textbf{Stabilizing ADC Readings:} At first, the ADC readings were too noisy to use reliably. 
    It was fixed by applying a moving average filter. This helped smooth out sudden changes 
    and made it easier to detect tones accurately.

    
    \item \textbf{Timing Calibration:} Detecting the correct duration of tones and silences was tricky. 
    Small changes in signal behavior could lead to dots being misread as dashes or vice versa. 
    The timing thresholds like \texttt{DOT\_DURATION} and \texttt{DASH\_DURATION}  were tested and adjusted
    multiple times until the system became reliable.
    
    \item \textbf{Error handing bug:} Initially, after detecting an error, the system would stop and wait for a reboot.
    This behavior was later improved to allow the system to continue receiving signal 
    while handling the error in parallel.  

      
\end{itemize}




\section{Conclusion}
The Morse code receiver project gave us valuable hands-on experience with microcontroller systems, 
analog signal processing, and real-time decoding. Throughout the project, 
we learned how to read and process analog signals using the ADC on the LPC4088 microcontroller, 
apply filtering techniques, and implement logic to accurately demodulate and decode Morse code 
into English text.
We successfully met the functional requirements, implemeted several extra features, ensured 
that our work is intuitive and user-friendly, and even aquired the unexpected experience
of 3D printing.

All the source code can by found on the GitHub repository:
\begin{center}
    \url{https://github.com/imanturewicz/Microcontrollers_4thSem}
\end{center}




\end{document}


