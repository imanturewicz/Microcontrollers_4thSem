\documentclass{cce2014-design}
\svnInfo $Id$

% Document details
\title{Design Brief: Morse Code Receiver System Group 4}
\author{
   Ignacy Manturewicz,
   Bradley Tabone,
   Aidan Cristina,
   Ben Okubamicael,
   
   }
\date{\svnMaxToday, Document v.\svnInfoMaxRevision}

\begin{document}

\maketitle

\abstract{%
This design brief outlines the development of a Morse code receiver using an ARM Cortex-M4 microcontroller. The project aims to decode Morse code signals 
effectively and display them, addressing the need for reliable low-bandwidth communication in [specific applications or scenarios].
This system will integrate real-time processing, user-friendly interfacing, and efficient use of hardware resources.
   }

\section{Introduction}
Write an introduction, explaining the purpose and background of this project.
Give a brief description of the game to be implemented.
References to be cited like so \cite{stroustrup2000}.
Also include an overview of this document.

\section{System Design}


\subsection{Hardware Configuration}

With regards to hardware, our Morse code receiver will be made up of 3 main components, a device, a breadboard and the ARM MK4 
microcontroller. The project begins with a device such as a laptop which will hold the Morse code audio signals. The audio signals
 are then outputted through the audio jack port of the computer and connected to the breadboard via jumper wire adapters. This 
 facilitates the physical connection between the two devices which allos the audio to be directly inputted into the breadboard's 
 circuitry where we will move onto the modification of the audio signals. Lastly, we connect jumper wires to the output pins of the breadboard 
 directly to the audio inputs of our microcontroller in order to continue with processing and translation of the audio signals to
 actual text which will be displayed on the LCD screen found within the microcontroller.

 \subsection{Software Configuration}

 With regards to software, we will be making use of Keil Uvision 5 which is used efficiently to interact directly with the 
 microcontroller. In this software we will write C code which will deal with the filtering/amplification of the audio signal obtained
 from the breadboard as well as code for translating the morse code into actual text which will then be outputted on the LCD screen.





\section{Management}
Include a time plan.
Show task dependencies. - Functionality dependencies ex. Testing depends on the functionality being done. How functions depend on one another.
How are these going to be managed?

\section{Closure}

\bibliographystyle{ieeetr}
\bibliography{references}

\end{document}
